%! TEX root = ../main.tex
\documentclass[main]{subfiles}

\begin{document}
\chapter{目的}

世界最高峰のレースであるFormula1(以後,F1)では,その構成要素はタイヤやエンジン,車体などに分けることができ,
その中でもタイヤ・エンジンについては,レギュレーションによって規定されており少なからず開発の余地はあるが,車体の空力特性がF1の車体を評価する上で最も重要な要素となっている.\cite{ref:f1-wind-tunnel}
車体の空力特性,すなわち
ストレートでは抵抗を抑え最高速度の最大化,
カーブではダウンフォースを最大化し最低速度を最大化すること,
また各カーブ間で最低速度から最高速度に達するまでの加速性能を向上させること
などが求められる.
また,商業的な側面からF1の運営団体は,
F1の車体の広報にできる乱流をできるだけ抑え後方の車もクリーンエアーを得られるようにするようなレギュレーションの設定
などさまざまな要素がある.

このような空力特性を向上させるため,風洞と呼ばれる巨大な設備を使用して車体に風を当てそれを調べる方法がしばしば取られる.
一方で,莫大な費用や電力を使用するため,各チーム間での格差是正のためこのような風洞を使用する時間はF1では制限されており,\cite{ref:f1-sporting-regulations}
それ以外の方法としてコンピュータを使用した流体シミュレーションが行われている.

今回は,このような流体シミュレーションを行うためのオープンソースソフトウェアであるFluidX3D
\footnote{
    \url{https://github.com/ProjectPhysX/FluidX3D}.
    FluidX3Dは研究・学習目的の場合には無料で使用できる.
}
を使用しコンピュータを用いた流体シミュレーションの手法の一つである格子ボルツマン法の理解と、
実施にさまざまなパーツについて、パラメータを自ら変更しつつ流体シミュレーションを行い、
その結果として得られる流体の流れ、抗力や浮力などの値などを評価することを目的とする.

\end{document}