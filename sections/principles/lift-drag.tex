一般に抵抗は物体の形状,流体の速度,粘度,圧縮率,速度に比例する.
物体の形状や,流体の粘度,圧縮率などによる抵抗の影響は簡単には表すことができず,
これらを含めて抵抗係数$Cd$として表し,抵抗力$D$は次のようにあらわされる.
\begin{equation}
    D = \frac{1}{2} \rho v^2 A Cd
\end{equation}
同様に,浮力$L$は浮力係数を$Cf$として次のように表される.
\begin{equation}
    L = \frac{1}{2} \rho v^2 A Cl
\end{equation}
ここで,$A$は流体の影響を受ける面積,$\rho$は流体の密度,$v$は流体の速度である.

一般に$Cd$,$Cl$は計算で求めることができず,実験によって求める必要がある.