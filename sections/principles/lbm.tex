\subsection{格子ボルツマン法の概要}
Lattice Boltzman Metod(格子ボルツマン法,以後LBM)は,さまざまなシミュレーションのパラメータの設定,物体の周りの流れの計算などができる流体シミュレーションの手法である.
Lattice Boltzman Methodと,Navier-Stokes方程式がどのように関係しているかは特に説明しない.
\footnote{
    次の書籍を参考にしてほしい.\cite{ref:kruger2017lattice}, \cite{ref:chapman1990mathematical}
}
以下は,LBMが何を解析しているかを簡単に説明するにとどめる\cite{ref:lbm}.

LBMは空間をカーテシアン格子状に分割し,各格子に$f_i$で表される密度分布関数(density distrubution functions, 以下DDFs)を導入する.
ここで添字$i$は,$f_i$が流れる方向の格子を指す.
この$f_i$を流体の分子のメゾスコピックな塊と見なすことができる.
この$f_i$が全て隣の格子に移動したのち,衝突や流体の流れを計算し,再び$f_i$を移動させることを繰り返すことで,流体の流れを計算する.

位置$\bm{x}$,時間$t$において,ある格子は隣接する格子からの流入を受ける.この時,隣接する格子は$\bm{x}-\bm{e}_i$と表さえる.
ここで,$\bm{e}_i$は後に\ref{sec:lbm-velocity-set}で定義するベクトルである.
流れ込むDDFsは,
\begin{equation}\label{eq:streaming}
    \ftemp(\bm{x}, t)=f_{i}(\bm{x}-\bm{e}_i, t).
\end{equation}
この$\ftemp$を用いて,密度・速度が計算される.
\begin{equation}\label{eq:rho}
    \rho(\bm{x}, t)=\sum_{i} \ftemp(\bm{x}, t)
\end{equation}
\begin{equation}\label{eq:velocity}
    \bm{u}(\bm{x}, t)=\frac{1}{\rho(\bm{x}, t)} \sum_{i} \ftemp(\bm{x}, t) \bm{e}_{i}
\end{equation}
さらに,$\ftemp$, $\rho$, $\bm{u}$を用いて,並行状態のDDFs$\feq$を計算する.
\begin{equation}\label{eq:equil-ddf}
    \feq(\rho(\bm{x}, t), \bm{u}(\bm{x}, t)) = w_i \rho(\bm{x}, t) \left(1 + \frac{\bm{e}_i \cdot \bm{u}(\bm{x}, t)}{c_s^2} + \frac{(\bm{e}_i \cdot \bm{u}(\bm{x}, t))^2}{2c_s^4} - \frac{\bm{u}(\bm{x}, t)^2}{2c_s^2} \right)
\end{equation}
ただし,$w_i$は重みであり後に$c_i$とともに\ref{sec:lbm-velocity-set}で定義する.また,$c_s$は格子の音速である.
最後に,$\feq$と$\ftemp$,これらから計算される衝突作用素$\Omega$を用いて,$\Delta t$時間後のDDFsを計算する.
\begin{equation}\label{eq:next-ddf}
    f_i(\bm{x}, t+\Delta t) = \ftemp(\bm{x}, t) - \Omega
\end{equation}
$\Omega$は具体的に\ref{sec:lbm-collision}で説明する.
これを繰り返すことで,流体の流れを計算する.

\subsection{初期条件}
初期条件は,$f_i(\bm{x}, t=0)$が必要となるが,
これは$\rho(\bm{x}, t=0)$と$\bm{u}(\bm{x}, t=0)$を定数として与え,
\begin{equation}\label{eq:initial-ddf}
    f_i(\bm{x}, t=0) = \feq(\rho(\bm{x}, t=0), \bm{u}(\bm{x}, t=0))
\end{equation}
とすることで与えられる.

\subsection{速度セット velocity set}\label{sec:lbm-velocity-set}
$\bm{e}_i$と,$w_i$の定義は以下の通りである.
これらはVelocity Setsと呼ばれ,一般に$d$を次元,$q$を速度の方向として$DdQq$として表される.
よく使われるものは,$D2Q9$,$D3Q15$,$D3Q13$,$D3Q19$,$D3Q27$である.
今回使用したのは,$D3Q19$であり,$\bm{e}_i$と$w_i$は以下の通りである.

\begin{landscape}
    \centering
    \begin{equation}
        w_i = \left[
            \frac{1}{3}, \frac{1}{18}, \frac{1}{18}, \frac{1}{18}, \frac{1}{18}, \frac{1}{18}, \frac{1}{18}, \frac{1}{36}, \frac{1}{36}, \frac{1}{36}, \frac{1}{36}, \frac{1}{36}, \frac{1}{36}, \frac{1}{36}, \frac{1}{36}, \frac{1}{36}, \frac{1}{36}
            \right]
    \end{equation}

    \begin{equation}
        \bm{e}_i =
        \left[
            \begin{array}[]{llllllllllllllllllll}
                0 & 1 & -1 & 0 & 0  & 0 &   & 0 1 & -1 & 1 & 1 & -1 & 0  & 0  & 1  & -1 & 1  & -1 & 0  & 0  \\
                0 & 0 & 0  & 1 & -1 & 0 & 0 & 1   & -1 & 0 & 0 & 1  & -1 & -1 & -1 & 1  & 0  & 0  & 1  & -1 \\
                0 & 0 & 0  & 0 & 0  & 0 & 1 & -1  & 0  & 0 & 1 & -1 & 1  & -1 & 0  & 0  & -1 & 1  & -1 & 1
            \end{array}
            \right]
    \end{equation}
\end{landscape}

\subsection{衝突作用素 collision operator}
衝突作用素$\Omega$はRelaxation Time$\tau$によって定義され,SRT,TRT,MRTなどRelaxationTimeの種類により複数ある.
今回は,最も簡単なSRTを用いて$\Omega$は,
\begin{equation}
    \Omega_i = \frac{\Delta t}{\tau} \left( \feq - \ftemp \right)
\end{equation}

\subsection{衝突について}\label{sec:lbm-collision}
LBMにおいて境界はEquilibrium Boundaries, Moving Bounce-back Boundaries, Non-moving Bounce-back Boundariesの3種類に分類される.
Equilibrium Boundariesの場合は,その境界を常に並行状態にあるとする.
\begin{equation}
    f_i(\bm{x}, t+\Delta t) =
    \begin{dcases}
        \ftemp(\bm{x}, t) - \Omega_i & \text{if $\bm{x}$ is a regular lattice point}              \\
        \feq(\rho_o, \bm{u}_0)       & \text{if $\bm{x}$ is a equilibrium boundary lattice point}
    \end{dcases}
\end{equation}
Non-moving Bounce-Back Boundariesの場合は,その境界における流れは反射するとする.
\begin{equation}
    f_{\text{temp}}(\vec{x}, t) :=
    \begin{dcases}
        f_A(\vec{x} - \vec{e}_i, t) & \text{if } (\vec{x} - \vec{e}_i) \text{ is a wall lattice point} \\
        f_iA(\vec{x}, t)            & \text{otherwise}
    \end{dcases}
\end{equation}

LBMの境界における衝突については,momentum exhange algorithm(運動量交換アルゴリズム)を用いて計算される.
DDFsと速度$\bm{c}_i=\bm{e}_i$の積の和は運動量に等しいから,$V$を格子の体積$V=(\Delta x)^3$として,
\begin{equation}\label{eq:momentum-exchange}
    \frac{p(\bm{x}, t)}{V} = \rho(\bm{x}, t) \bm{u}(\bm{x}, t) = \sum_i f_i(\bm{x}, t) \bm{e}_i
\end{equation}
\begin{equation}\label{eq:force}
    f = \frac{p(\bm{x}, t)}{V \Delta t} = \frac{1}{\Delta t} \sum_i f_i(\bm{x}, t) \bm{e}_i
\end{equation}