
前述の通り,流体シミュレーションのソフトウェアとしてFluidX3Dを使用する.
FluidX3DはC++で書かれたCFDソフトウェアで,格子ボルツマン法を用いており,
VRAMやRAMの容量をそれほど必要とせず,
さまざまなモデルを組み込み,流体の速度や大きさなどを指定して流体の流れの可視化,撮影,
抗力や浮力の計算などを行うことができる.

実際にシミュレーションをした環境は以下の通りである.

\begin{quote}
    \begin{itemize}
        \item CPU: Intel Core i5-12600KF
        \item GPU: NVIDIA GeForce RTX 3060 12GB
        \item RAM: 46GB
    \end{itemize}
\end{quote}