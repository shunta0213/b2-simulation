%! TEX root = ../main.tex
\documentclass[main]{subfiles}

\begin{document}
\chapter{おわりに}
本実験を通して,FluidX3Dを用いて簡単な物体の周りの気流の流れの可視化、抗力、揚力の測定を目的として流体シミュレーションを行い、
LBM法の基本的な考え方を理解でき,
FluidX3Dを用いてLBMを使用した流体シミュレーションをいくつかのモデルについて行った.
また,そのシミュレーションの結果をもとに物体の周りの気流の流れ,
得られた抵抗係数について簡単な考察をした.

今後の課題として,流体の基礎方程式であるNavier-Stokes方程式とLBMの関係性の理解や,
LBMを実装する方法,
LBMをGPUなどの並列計算環境で実行し高速化・計算量を減らす工夫の理解
などが挙げられる.
さらに,Ahmed Bodyでは$Cd$の値が小さく周期的に変化していると読み取ることができたが,
この理由についても調査する必要がある.

\end{document}